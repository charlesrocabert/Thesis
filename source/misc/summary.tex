
\chapternonum{R\'{e}sum\'{e}}

\section*{(en fran\c cais)}

Variation et s\'{e}lection sont au coeur de l'\'{e}volution Darwinienne. Cependant, ces deux m\'{e}canismes d\'{e}pendent de processus eux-m\^{e}me fa\c conn\'{e}s par l'\'{e}volution. Chez les micro-organismes, qui font face \`a des environnements souvent variables, ces propri\'{e}t\'{e}s adaptatives sont particuli\`erement bien exploit\'{e}es, comme le d\'{e}montrent de nombreuses exp\'{e}riences en laboratoire. Chez ses organismes, l'\'{e}volution semble donc avoir optimis\'{e} sa propre capacit\'{e} \`a \'{e}voluer, un processus que nous nommons \'{e}volution de l'\'{e}volution (EvoEvo).
La notion d'\'{e}volution de l'\'{e}volution englobe de nombreux concepts th\'{e}oriques, tels que la variabilit\'{e}, l'\'{e}volvabilit\'{e}, la robustesse ou encore la capacit\'{e} de l'\'{e}volution \`a innover (open-endedness). Ces propri\'{e}t\'{e}s \'{e}volutives des micro-organismes, et plus g\'{e}n\'{e}ralement de tous les organismes vivants, sont soup\c conn\'{e}es d'agir \`a tous les niveaux d'organisation biologique, en interaction ou en conflit, avec des cons\'{e}quences souvent complexes et contre-intuitives. Ainsi, comprendre l'\'{e}volution de l'\'{e}volution implique l'\'{e}tude de la trajectoire \'{e}volutive de micro-organismes -- r\'{e}els ou virtuels -- et ce \`a diff\'{e}rents niveaux d'organisation (g\'{e}nome, interactome, population, ...).

L'objectif de ce travail de th\`ese a \'{e}t\'{e} de d\'{e}velopper et d'\'{e}tudier des mod\`eles math\'{e}matiques et num\'{e}riques afin de lever le voile sur certains aspects de l'\'{e}volution de l'\'{e}volution. Ce travail multidisciplinaire, car impliquant des collaborations avec des biologistes exp\'{e}rimentateur$\cdot$rice$\cdot$s, des bio-informaticien$\cdot$ne$\cdot$s et des math\'{e}maticien$\cdot$ne$\cdot$s, s'est divis\'{e} en deux parties distinctes, mais compl\'{e}mentaires par leurs approches : \textbf{(i)} l'extension d'un mod\`ele historique en g\'{e}n\'{e}tique des populations -- le mod\`ele g\'{e}om\'{e}trique de Fisher -- afin d'\'{e}tudier l'\'{e}volution du bruit ph\'{e}notypique en s\'{e}lection directionnelle, et \textbf{(ii)} le d\'{e}veloppement d'un mod\`ele d'\'{e}volution \textit{in silico} multi-\'{e}chelles permettant une \'{e}tude plus approfondie de l'\'{e}volution de l'\'{e}volution.
Dans un premier temps, gr\^{a}ce \`a une version \'{e}tendue du mod\`ele de Fisher, nous avons montr\'{e} qu'un bruit corr\'{e}l\'{e} sur diff\'{e}rents caract\`eres ph\'{e}notypiques \'{e}volue sous s\'{e}lection directionnelle vers une forme bien particuli\`ere permettant de compenser en grande partie le co\^{u}t de la ``complexit\'{e} ph\'{e}notypique'', qui limite habituellement et fortement les chances de fixer des mutations favorables lorsque le nombre de caract\`eres sous s\'{e}lection est grand. Ces r\'{e}sultats prometteurs d\'{e}montrent l'importance et l'avantage s\'{e}lectif du bruit ph\'{e}notypique en s\'{e}lection directionnelle, et devrait susciter de nouveaux travaux de recherche, en collaboration avec des biologistes exp\'{e}rimentateur$\cdot$rice$\cdot$s.
Dans un deuxi\`eme temps, gr\^{a}ce au d\'{e}veloppement d'un mod\`ele d'\'{e}volution exp\'{e}rimentale \textit{in silico} multi-\'{e}chelles, nous avons pu reproduire virtuellement des approches exp\'{e}rimentales \textit{in vivo} -- notamment l'exp\'{e}rience d'\'{e}volution \`a long terme (LTEE) -- et ainsi contribu\'{e} \`a comprendre les ph\'{e}nom\`enes de construction de niche et d'\'{e}mergence d'un cross-feeding stable, pr\'{e}misses \`a la diversification et \`a la sp\'{e}ciation bact\'{e}rienne.
Ce mod\`ele d'\'{e}volution \textit{in silico} nous a \'{e}galement permis de nous interroger sur l'\'{e}mergence et l'\'{e}volution de la r\'{e}gulation g\'{e}n\'{e}tique, comme solution au maintien de l'\'{e}conomie \'{e}nerg\'{e}tique de la cellule. Nos r\'{e}sultats, encore pr\'{e}liminaires, confirment l'id\'{e}e que le r\^{o}le de la r\'{e}gulation g\'{e}n\'{e}tique n'est pas d'ajuster le m\'{e}tabolisme aux conditions environnementales, mais plut\^ot d'assurer l'\'{e}conomie \'{e}nerg\'{e}tique interne et la survie de la cellule, ind\'{e}pendamment des variations environnementales. Ces premiers r\'{e}sultats sugg\`erent \'{e}galement que la structuration des g\'{e}nomes bact\'{e}riens est fortement influenc\'{e}e par les contraintes \'{e}nerg\'{e}tiques internes.

Cette th\`ese a \'{e}t\'{e} financ\'{e}e par le projet europ\'{e}en EvoEvo (FP7-ICT-610427), gr\^{a}ce \`a la commission europ\'{e}enne.

\section*{(in english)}

Variation and selection are the two core processes of Darwinian Evolution. Yet, both are directly regulated by many processes that are themselves products of evolution. Microorganisms efficiently exploit this ability to dynamically adapt to new conditions. Thus, evolution seems to have optimized its own ability to evolve, as a primary means to react to environmental changes. We call this process evolution of evolution (EvoEvo). EvoEvo covers several aspects of evolution, encompassing major concepts such variability, evolvability, robustness, and open-endedness. Those phenomena are known to affect all levels of organization in bacterial populations. Indeed, understanding EvoEvo requires to study organisms experiencing evolution, and to decipher the evolutive interactions between all the components of the biological system of interest (genomes, biochemical networks, populations, ...).
The objective of this thesis was to develop and exploit mathematical and numerical models to tackle different aspects of EvoEvo, in order to produce new knowledge on this topic, in collaboration with partners from diverse fields, including experimental biology, bioinformatics, mathematics and also theoretical and applied informatics. To this aim, we followed two complementary approaches: \textbf{(i)} a population genetics approach to study the evolution of phenotypic noise in directional selection, by extending Fisher's geometric model of adaptation, and \textbf{(ii)} a digital genetics approach to study multi-level evolution. This work was funded by the EvoEvo project, under the European Commission (FP7-ICT-610427).

